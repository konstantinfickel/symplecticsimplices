\documentclass[../SymplecticSimplices.tex]{subfiles}

\begin{document}
\section{Introduction}

In this bachelor thesis, a classification of symplectic simplices will be provided. Possible applications of those results could be in the field of piecewise linear symplectic geometry, about which not many publications exist:

In 1998 Bernhard Gratza introduced the term of piecewise-linear symplectic geometry in his doctoral thesis \cite{gratza} as an approach to solve finite-dimensional symplectic problems ``turning them into combinatorical ones''. As a first result he proved that compactly supported Hamiltonian diffeomorphisms can be approximated by piecewise-linear symplectic maps in the \( \mathcal{C}^0 \)-norm.

Another, more recent, publication is the doctoral thesis by Julie Distexhe \cite{distexhe} submitted in 2019, in which she considered the construction of symplectic triangulations for smooth manifolds. Additionally, she proved a theorem triangulating smooth volume forms and provided a symplectic analogy of Thurstons jiggling lemma.

When triangulating a higher-dimensional object is decomposed into simplices. One interesting question that arises about those simplices is how to classify them up to linear symplectomorphisms.

Something similar was done by Henriques and Pak proved in their unpublished paper \cite{henri}: They proved that there exists a volume-preserving, piecewise-linear map between two convex polytopes with identical volume.

In an unpublished note \cite{cieliebak} by Cieliebak and Hofer it was shown that in \( \mathbb{R}^4 \) two \( 4 \)-simplices are equal up to affine symplectomorphism if and only if the area of their two-dimensional subsimplices coincides and gave a formula to construct a simplex for some given two-dimensional symplectic areas.

Those ideas will be generalized to \( 2n \)-simplices in \( \mathbb{R}^{2n} \) in this thesis: In the second section, a short introduction to linear symplectic geometry is given. Then, the notion of symplectic volume is introduced and some general statements about it are shown. In the fourth and main chapter the existence and uniqueness up to linear symplectomorphism of a \( 2n \)-simplex for given areas of its 2-dimensional subsimplices will be proven.
\end{document}

